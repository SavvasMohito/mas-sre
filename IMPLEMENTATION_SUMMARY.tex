% Options for packages loaded elsewhere
% Options for packages loaded elsewhere
\PassOptionsToPackage{unicode}{hyperref}
\PassOptionsToPackage{hyphens}{url}
\PassOptionsToPackage{dvipsnames,svgnames,x11names}{xcolor}
%
\documentclass[
  a4paper,
]{report}
\usepackage{xcolor}
\usepackage[lmargin=1in,rmargin=1in,tmargin=1in,bmargin=1in]{geometry}
\usepackage{amsmath,amssymb}
\setcounter{secnumdepth}{5}
\usepackage{iftex}
\ifPDFTeX
  \usepackage[T1]{fontenc}
  \usepackage[utf8]{inputenc}
  \usepackage{textcomp} % provide euro and other symbols
\else % if luatex or xetex
  \usepackage{unicode-math} % this also loads fontspec
  \defaultfontfeatures{Scale=MatchLowercase}
  \defaultfontfeatures[\rmfamily]{Ligatures=TeX,Scale=1}
\fi
\usepackage{lmodern}
\ifPDFTeX\else
  % xetex/luatex font selection
\fi
% Use upquote if available, for straight quotes in verbatim environments
\IfFileExists{upquote.sty}{\usepackage{upquote}}{}
\IfFileExists{microtype.sty}{% use microtype if available
  \usepackage[]{microtype}
  \UseMicrotypeSet[protrusion]{basicmath} % disable protrusion for tt fonts
}{}
\makeatletter
\@ifundefined{KOMAClassName}{% if non-KOMA class
  \IfFileExists{parskip.sty}{%
    \usepackage{parskip}
  }{% else
    \setlength{\parindent}{0pt}
    \setlength{\parskip}{6pt plus 2pt minus 1pt}}
}{% if KOMA class
  \KOMAoptions{parskip=half}}
\makeatother
% Make \paragraph and \subparagraph free-standing
\makeatletter
\ifx\paragraph\undefined\else
  \let\oldparagraph\paragraph
  \renewcommand{\paragraph}{
    \@ifstar
      \xxxParagraphStar
      \xxxParagraphNoStar
  }
  \newcommand{\xxxParagraphStar}[1]{\oldparagraph*{#1}\mbox{}}
  \newcommand{\xxxParagraphNoStar}[1]{\oldparagraph{#1}\mbox{}}
\fi
\ifx\subparagraph\undefined\else
  \let\oldsubparagraph\subparagraph
  \renewcommand{\subparagraph}{
    \@ifstar
      \xxxSubParagraphStar
      \xxxSubParagraphNoStar
  }
  \newcommand{\xxxSubParagraphStar}[1]{\oldsubparagraph*{#1}\mbox{}}
  \newcommand{\xxxSubParagraphNoStar}[1]{\oldsubparagraph{#1}\mbox{}}
\fi
\makeatother

\usepackage{color}
\usepackage{fancyvrb}
\newcommand{\VerbBar}{|}
\newcommand{\VERB}{\Verb[commandchars=\\\{\}]}
\DefineVerbatimEnvironment{Highlighting}{Verbatim}{commandchars=\\\{\}}
% Add ',fontsize=\small' for more characters per line
\usepackage{framed}
\definecolor{shadecolor}{RGB}{241,243,245}
\newenvironment{Shaded}{\begin{snugshade}}{\end{snugshade}}
\newcommand{\AlertTok}[1]{\textcolor[rgb]{0.68,0.00,0.00}{#1}}
\newcommand{\AnnotationTok}[1]{\textcolor[rgb]{0.37,0.37,0.37}{#1}}
\newcommand{\AttributeTok}[1]{\textcolor[rgb]{0.40,0.45,0.13}{#1}}
\newcommand{\BaseNTok}[1]{\textcolor[rgb]{0.68,0.00,0.00}{#1}}
\newcommand{\BuiltInTok}[1]{\textcolor[rgb]{0.00,0.23,0.31}{#1}}
\newcommand{\CharTok}[1]{\textcolor[rgb]{0.13,0.47,0.30}{#1}}
\newcommand{\CommentTok}[1]{\textcolor[rgb]{0.37,0.37,0.37}{#1}}
\newcommand{\CommentVarTok}[1]{\textcolor[rgb]{0.37,0.37,0.37}{\textit{#1}}}
\newcommand{\ConstantTok}[1]{\textcolor[rgb]{0.56,0.35,0.01}{#1}}
\newcommand{\ControlFlowTok}[1]{\textcolor[rgb]{0.00,0.23,0.31}{\textbf{#1}}}
\newcommand{\DataTypeTok}[1]{\textcolor[rgb]{0.68,0.00,0.00}{#1}}
\newcommand{\DecValTok}[1]{\textcolor[rgb]{0.68,0.00,0.00}{#1}}
\newcommand{\DocumentationTok}[1]{\textcolor[rgb]{0.37,0.37,0.37}{\textit{#1}}}
\newcommand{\ErrorTok}[1]{\textcolor[rgb]{0.68,0.00,0.00}{#1}}
\newcommand{\ExtensionTok}[1]{\textcolor[rgb]{0.00,0.23,0.31}{#1}}
\newcommand{\FloatTok}[1]{\textcolor[rgb]{0.68,0.00,0.00}{#1}}
\newcommand{\FunctionTok}[1]{\textcolor[rgb]{0.28,0.35,0.67}{#1}}
\newcommand{\ImportTok}[1]{\textcolor[rgb]{0.00,0.46,0.62}{#1}}
\newcommand{\InformationTok}[1]{\textcolor[rgb]{0.37,0.37,0.37}{#1}}
\newcommand{\KeywordTok}[1]{\textcolor[rgb]{0.00,0.23,0.31}{\textbf{#1}}}
\newcommand{\NormalTok}[1]{\textcolor[rgb]{0.00,0.23,0.31}{#1}}
\newcommand{\OperatorTok}[1]{\textcolor[rgb]{0.37,0.37,0.37}{#1}}
\newcommand{\OtherTok}[1]{\textcolor[rgb]{0.00,0.23,0.31}{#1}}
\newcommand{\PreprocessorTok}[1]{\textcolor[rgb]{0.68,0.00,0.00}{#1}}
\newcommand{\RegionMarkerTok}[1]{\textcolor[rgb]{0.00,0.23,0.31}{#1}}
\newcommand{\SpecialCharTok}[1]{\textcolor[rgb]{0.37,0.37,0.37}{#1}}
\newcommand{\SpecialStringTok}[1]{\textcolor[rgb]{0.13,0.47,0.30}{#1}}
\newcommand{\StringTok}[1]{\textcolor[rgb]{0.13,0.47,0.30}{#1}}
\newcommand{\VariableTok}[1]{\textcolor[rgb]{0.07,0.07,0.07}{#1}}
\newcommand{\VerbatimStringTok}[1]{\textcolor[rgb]{0.13,0.47,0.30}{#1}}
\newcommand{\WarningTok}[1]{\textcolor[rgb]{0.37,0.37,0.37}{\textit{#1}}}

\usepackage{longtable,booktabs,array}
\usepackage{calc} % for calculating minipage widths
% Correct order of tables after \paragraph or \subparagraph
\usepackage{etoolbox}
\makeatletter
\patchcmd\longtable{\par}{\if@noskipsec\mbox{}\fi\par}{}{}
\makeatother
% Allow footnotes in longtable head/foot
\IfFileExists{footnotehyper.sty}{\usepackage{footnotehyper}}{\usepackage{footnote}}
\makesavenoteenv{longtable}
\usepackage{graphicx}
\makeatletter
\newsavebox\pandoc@box
\newcommand*\pandocbounded[1]{% scales image to fit in text height/width
  \sbox\pandoc@box{#1}%
  \Gscale@div\@tempa{\textheight}{\dimexpr\ht\pandoc@box+\dp\pandoc@box\relax}%
  \Gscale@div\@tempb{\linewidth}{\wd\pandoc@box}%
  \ifdim\@tempb\p@<\@tempa\p@\let\@tempa\@tempb\fi% select the smaller of both
  \ifdim\@tempa\p@<\p@\scalebox{\@tempa}{\usebox\pandoc@box}%
  \else\usebox{\pandoc@box}%
  \fi%
}
% Set default figure placement to htbp
\def\fps@figure{htbp}
\makeatother





\setlength{\emergencystretch}{3em} % prevent overfull lines

\providecommand{\tightlist}{%
  \setlength{\itemsep}{0pt}\setlength{\parskip}{0pt}}



 


\usepackage{fvextra}
\DefineVerbatimEnvironment{Highlighting}{Verbatim}{breaklines,commandchars=\\\{\}}
\makeatletter
\@ifpackageloaded{caption}{}{\usepackage{caption}}
\AtBeginDocument{%
\ifdefined\contentsname
  \renewcommand*\contentsname{Table of contents}
\else
  \newcommand\contentsname{Table of contents}
\fi
\ifdefined\listfigurename
  \renewcommand*\listfigurename{List of Figures}
\else
  \newcommand\listfigurename{List of Figures}
\fi
\ifdefined\listtablename
  \renewcommand*\listtablename{List of Tables}
\else
  \newcommand\listtablename{List of Tables}
\fi
\ifdefined\figurename
  \renewcommand*\figurename{Figure}
\else
  \newcommand\figurename{Figure}
\fi
\ifdefined\tablename
  \renewcommand*\tablename{Table}
\else
  \newcommand\tablename{Table}
\fi
}
\@ifpackageloaded{float}{}{\usepackage{float}}
\floatstyle{ruled}
\@ifundefined{c@chapter}{\newfloat{codelisting}{h}{lop}}{\newfloat{codelisting}{h}{lop}[chapter]}
\floatname{codelisting}{Listing}
\newcommand*\listoflistings{\listof{codelisting}{List of Listings}}
\makeatother
\makeatletter
\makeatother
\makeatletter
\@ifpackageloaded{caption}{}{\usepackage{caption}}
\@ifpackageloaded{subcaption}{}{\usepackage{subcaption}}
\makeatother
\usepackage{bookmark}
\IfFileExists{xurl.sty}{\usepackage{xurl}}{} % add URL line breaks if available
\urlstyle{same}
\hypersetup{
  colorlinks=true,
  linkcolor={blue},
  filecolor={Maroon},
  citecolor={Blue},
  urlcolor={Blue},
  pdfcreator={LaTeX via pandoc}}


\author{}
\date{}
\begin{document}

\renewcommand*\contentsname{Table of contents}
{
\hypersetup{linkcolor=}
\setcounter{tocdepth}{2}
\tableofcontents
}

\chapter{Implementation Summary}\label{implementation-summary}

\section{Multi-Agent Security Requirements Generation
System}\label{multi-agent-security-requirements-generation-system}

\textbf{Date}: October 2025 \textbf{Framework}: CrewAI Flows
\textbf{Status}: Implementation Complete ✅

\begin{center}\rule{0.5\linewidth}{0.5pt}\end{center}

\section{System Overview}\label{system-overview}

This implementation delivers a fully functional multi-agent system that
transforms high-level product requirements into comprehensive,
standards-aligned security requirements using CrewAI's Flow
orchestration.

\subsection{Architecture}\label{architecture}

The system consists of 5 specialized AI agents working in sequence with
a self-evaluation loop:

\begin{verbatim}
Input (Text File)
    ↓
Requirements Analysis Agent
    ↓
┌──────────────────────────────────────┐
│  Domain Security Agent               │
│  LLM Security Specialist             │
│  Compliance Agent                    │
└──────────────────────────────────────┘
    ↓
Validation Agent
    ↓
[Self-Evaluation Decision]
    ↓
Output (JSON + Markdown)
\end{verbatim}

\begin{center}\rule{0.5\linewidth}{0.5pt}\end{center}

\section{Components Implemented}\label{components-implemented}

\subsection{1. Core Agents (5 Crews)}\label{core-agents-5-crews}

\subsubsection{\texorpdfstring{\textbf{Requirements Analysis
Crew}}{Requirements Analysis Crew}}\label{requirements-analysis-crew}

\begin{itemize}
\tightlist
\item
  Location:
  \texttt{src/security\_requirements\_system/crews/requirements\_analysis\_crew/}
\item
  Purpose: Parse high-level inputs and identify security implications
\item
  Output: Structured analysis of business features, data types, user
  roles, integrations, and security concerns
\end{itemize}

\subsubsection{\texorpdfstring{\textbf{Domain Security
Crew}}{Domain Security Crew}}\label{domain-security-crew}

\begin{itemize}
\tightlist
\item
  Location:
  \texttt{src/security\_requirements\_system/crews/domain\_security\_crew/}
\item
  Purpose: Map requirements to OWASP, NIST, ISO 27001 controls
\item
  Tools: Weaviate vector database query tool
\item
  Output: Mapped security controls with implementation guidance
\end{itemize}

\subsubsection{\texorpdfstring{\textbf{LLM Security Specialist
Crew}}{LLM Security Specialist Crew}}\label{llm-security-specialist-crew}

\begin{itemize}
\tightlist
\item
  Location:
  \texttt{src/security\_requirements\_system/crews/llm\_security\_crew/}
\item
  Purpose: Identify AI/ML components and add specialized controls
\item
  Threats Covered: Prompt injection, data leakage, model poisoning,
  adversarial attacks
\item
  Output: AI-specific security requirements when applicable
\end{itemize}

\subsubsection{\texorpdfstring{\textbf{Compliance
Crew}}{Compliance Crew}}\label{compliance-crew}

\begin{itemize}
\tightlist
\item
  Location:
  \texttt{src/security\_requirements\_system/crews/compliance\_crew/}
\item
  Purpose: Identify regulatory requirements (GDPR, HIPAA, PCI-DSS, etc.)
\item
  Output: Compliance controls, data handling rules, audit requirements
\end{itemize}

\subsubsection{\texorpdfstring{\textbf{Validation
Crew}}{Validation Crew}}\label{validation-crew}

\begin{itemize}
\tightlist
\item
  Location:
  \texttt{src/security\_requirements\_system/crews/validation\_crew/}
\item
  Purpose: Self-evaluation on 5 dimensions (completeness, consistency,
  correctness, implementability, alignment)
\item
  Output: Validation score (0-1), pass/fail, gaps, conflicts,
  recommendations
\end{itemize}

\subsection{2. Flow Orchestration}\label{flow-orchestration}

\textbf{Main Flow}: \texttt{src/security\_requirements\_system/main.py}

\begin{itemize}
\tightlist
\item
  Implements \texttt{SecurityRequirementsFlow} class extending CrewAI
  Flow
\item
  Self-evaluation loop with max 3 iterations
\item
  Validation threshold: 0.8
\item
  Automatic feedback integration on failed validation
\item
  Generates both JSON and Markdown outputs
\end{itemize}

\subsection{3. Vector Database
Integration}\label{vector-database-integration}

\textbf{Weaviate Tools}: - \texttt{weaviate\_tool.py}: Custom CrewAI
tool for querying security standards - \texttt{weaviate\_setup.py}:
Schema initialization and data ingestion utilities

\textbf{Features}: - Semantic search over security controls - Filter by
standard (OWASP/NIST/ISO27001) - OpenAI text-embedding-3-small for
vectorization - Supports local Weaviate (Docker) or cloud deployment

\subsection{4. Security Standards Data}\label{security-standards-data}

\textbf{Preparation Scripts}: - \texttt{prepare\_owasp\_asvs.py}: OWASP
ASVS controls (17 sample controls) - \texttt{prepare\_nist.py}: NIST
Cybersecurity Framework (24 sample controls) -
\texttt{prepare\_iso27001.py}: ISO 27001:2022 Annex A (26 sample
controls)

\textbf{Total}: 67 security controls prepared across 3 standards

\subsection{5. Sample Inputs}\label{sample-inputs}

Three ready-to-use examples: 1. \textbf{Task Management App}
(\texttt{inputs/requirements.txt}) - Default 2. \textbf{E-Commerce
Platform} (\texttt{inputs/sample\_ecommerce.txt}) 3.
\textbf{Telemedicine Platform} (\texttt{inputs/sample\_healthcare.txt})

\subsection{6. Helper Scripts}\label{helper-scripts}

\begin{itemize}
\tightlist
\item
  \texttt{setup.sh}: Complete automated setup
\item
  \texttt{run\_example.sh}: Easy execution of different examples
\item
  \texttt{docker-compose.yml}: Weaviate database setup
\end{itemize}

\begin{center}\rule{0.5\linewidth}{0.5pt}\end{center}

\section{File Structure}\label{file-structure}

\begin{verbatim}
security_requirements_system/
├── src/
│   └── security_requirements_system/
│       ├── crews/                         # 5 agent crews
│       │   ├── requirements_analysis_crew/
│       │   ├── domain_security_crew/
│       │   ├── llm_security_crew/
│       │   ├── compliance_crew/
│       │   └── validation_crew/
│       ├── tools/                         # Custom tools
│       │   ├── weaviate_tool.py
│       │   └── weaviate_setup.py
│       ├── data/                          # Standards data
│       │   ├── prepared/                  # JSON files
│       │   ├── prepare_owasp_asvs.py
│       │   ├── prepare_nist.py
│       │   └── prepare_iso27001.py
│       └── main.py                        # Flow orchestration
├── inputs/                                # Sample requirements
│   ├── requirements.txt
│   ├── sample_ecommerce.txt
│   └── sample_healthcare.txt
├── outputs/                               # Generated results
├── docker-compose.yml
├── setup.sh
├── run_example.sh
├── config.yaml
├── env.template
├── README.md
├── QUICKSTART.md
└── pyproject.toml
\end{verbatim}

\begin{center}\rule{0.5\linewidth}{0.5pt}\end{center}

\section{Usage}\label{usage}

\subsection{Quick Start}\label{quick-start}

\begin{Shaded}
\begin{Highlighting}[]
\CommentTok{\# 1. Setup (one{-}time)}
\ExtensionTok{./setup.sh}

\CommentTok{\# 2. Run with default input}
\ExtensionTok{crewai}\NormalTok{ run}

\CommentTok{\# 3. Run with specific example}
\ExtensionTok{./run\_example.sh}\NormalTok{ ecommerce}
\end{Highlighting}
\end{Shaded}

\subsection{Manual Steps}\label{manual-steps}

\begin{Shaded}
\begin{Highlighting}[]
\CommentTok{\# 1. Configure environment}
\FunctionTok{cp}\NormalTok{ env.template .env}
\CommentTok{\# Edit .env with OPENAI\_API\_KEY}

\CommentTok{\# 2. Install dependencies}
\ExtensionTok{crewai}\NormalTok{ install}

\CommentTok{\# 3. Start Weaviate}
\ExtensionTok{docker{-}compose}\NormalTok{ up }\AttributeTok{{-}d}

\CommentTok{\# 4. Prepare and ingest data}
\ExtensionTok{python} \AttributeTok{{-}m}\NormalTok{ security\_requirements\_system.data.prepare\_owasp\_asvs}
\ExtensionTok{python} \AttributeTok{{-}m}\NormalTok{ security\_requirements\_system.data.prepare\_nist}
\ExtensionTok{python} \AttributeTok{{-}m}\NormalTok{ security\_requirements\_system.data.prepare\_iso27001}
\ExtensionTok{python} \AttributeTok{{-}m}\NormalTok{ security\_requirements\_system.tools.weaviate\_setup}

\CommentTok{\# 5. Run the system}
\VariableTok{INPUT\_FILE}\OperatorTok{=}\NormalTok{inputs/sample\_ecommerce.txt }\ExtensionTok{crewai}\NormalTok{ run}
\end{Highlighting}
\end{Shaded}

\begin{center}\rule{0.5\linewidth}{0.5pt}\end{center}

\section{Output Format}\label{output-format}

\subsection{\texorpdfstring{JSON Output
(\texttt{outputs/security\_requirements.json})}{JSON Output (outputs/security\_requirements.json)}}\label{json-output-outputssecurity_requirements.json}

\begin{Shaded}
\begin{Highlighting}[]
\FunctionTok{\{}
  \DataTypeTok{"metadata"}\FunctionTok{:} \FunctionTok{\{}
    \DataTypeTok{"validation\_score"}\FunctionTok{:} \FloatTok{0.85}\FunctionTok{,}
    \DataTypeTok{"validation\_passed"}\FunctionTok{:} \KeywordTok{true}\FunctionTok{,}
    \DataTypeTok{"iterations"}\FunctionTok{:} \DecValTok{1}
  \FunctionTok{\},}
  \DataTypeTok{"original\_requirements"}\FunctionTok{:} \StringTok{"..."}\FunctionTok{,}
  \DataTypeTok{"requirements\_analysis"}\FunctionTok{:} \StringTok{"..."}\FunctionTok{,}
  \DataTypeTok{"security\_controls"}\FunctionTok{:} \StringTok{"..."}\FunctionTok{,}
  \DataTypeTok{"ai\_ml\_security"}\FunctionTok{:} \StringTok{"..."}\FunctionTok{,}
  \DataTypeTok{"compliance\_requirements"}\FunctionTok{:} \StringTok{"..."}\FunctionTok{,}
  \DataTypeTok{"validation\_report"}\FunctionTok{:} \StringTok{"..."}
\FunctionTok{\}}
\end{Highlighting}
\end{Shaded}

\subsection{\texorpdfstring{Markdown Output
(\texttt{outputs/security\_requirements.md})}{Markdown Output (outputs/security\_requirements.md)}}\label{markdown-output-outputssecurity_requirements.md}

Human-readable summary with all sections formatted for easy review.

\begin{center}\rule{0.5\linewidth}{0.5pt}\end{center}

\section{Technical Specifications}\label{technical-specifications}

\subsection{Dependencies}\label{dependencies}

\begin{itemize}
\tightlist
\item
  \textbf{crewai{[}tools{]}} \textgreater= 0.157.0: Multi-agent
  orchestration
\item
  \textbf{weaviate-client} \textgreater= 4.0.0: Vector database
\item
  \textbf{openai} \textgreater= 1.0.0: LLM and embeddings
\item
  \textbf{python-dotenv} \textgreater= 1.0.0: Environment management
\end{itemize}

\subsection{Configuration}\label{configuration}

\begin{itemize}
\tightlist
\item
  \textbf{Max Iterations}: 3 (configurable in \texttt{main.py})
\item
  \textbf{Validation Threshold}: 0.8 (configurable in \texttt{main.py})
\item
  \textbf{LLM Model}: OpenAI GPT-4o-mini (default)
\item
  \textbf{Embedding Model}: text-embedding-3-small
\end{itemize}

\subsection{System Requirements}\label{system-requirements}

\begin{itemize}
\tightlist
\item
  Python 3.10-3.13
\item
  Docker for Weaviate
\item
  OpenAI API access
\item
  \textasciitilde2GB disk space
\end{itemize}

\begin{center}\rule{0.5\linewidth}{0.5pt}\end{center}

\section{Key Features Implemented}\label{key-features-implemented}

✅ \textbf{5-Agent System} as specified in thesis proposal ✅
\textbf{Self-Evaluation Loop} with iterative refinement ✅
\textbf{Vector Database} for semantic search over standards ✅
\textbf{Standards Coverage}: OWASP, NIST, ISO 27001 ✅ \textbf{AI/ML
Security} specialist handling ✅ \textbf{Compliance Assessment} for
GDPR, HIPAA, PCI-DSS ✅ \textbf{Validation Framework} with 5-dimensional
scoring ✅ \textbf{Sample Data} for 3 real-world scenarios ✅
\textbf{Automated Setup} scripts ✅ \textbf{Comprehensive Documentation}

\begin{center}\rule{0.5\linewidth}{0.5pt}\end{center}

\section{Extensibility}\label{extensibility}

\subsection{Adding New Security
Standards}\label{adding-new-security-standards}

\begin{enumerate}
\def\labelenumi{\arabic{enumi}.}
\item
  Create preparation script:

\begin{Shaded}
\begin{Highlighting}[]
\CommentTok{\# src/security\_requirements\_system/data/prepare\_custom.py}
\KeywordTok{def}\NormalTok{ prepare\_custom\_standard():}
\NormalTok{    controls }\OperatorTok{=}\NormalTok{ [\{}\StringTok{"standard\_name"}\NormalTok{: }\StringTok{"Custom"}\NormalTok{, ...\}]}
    \CommentTok{\# Save to prepared/custom.json}
\end{Highlighting}
\end{Shaded}
\item
  Run preparation and re-ingest:

\begin{Shaded}
\begin{Highlighting}[]
\ExtensionTok{python} \AttributeTok{{-}m}\NormalTok{ security\_requirements\_system.data.prepare\_custom}
\ExtensionTok{python} \AttributeTok{{-}m}\NormalTok{ security\_requirements\_system.tools.weaviate\_setup}
\end{Highlighting}
\end{Shaded}
\end{enumerate}

\subsection{Customizing Agents}\label{customizing-agents}

Edit YAML configurations in each crew's \texttt{config/} directory: -
\texttt{agents.yaml}: Modify roles, goals, backstories -
\texttt{tasks.yaml}: Adjust task descriptions, expected outputs

\subsection{Adjusting Validation}\label{adjusting-validation}

Modify in \texttt{main.py}:

\begin{Shaded}
\begin{Highlighting}[]
\NormalTok{MAX\_ITERATIONS }\OperatorTok{=} \DecValTok{5}  \CommentTok{\# More refinement loops}
\NormalTok{VALIDATION\_THRESHOLD }\OperatorTok{=} \FloatTok{0.9}  \CommentTok{\# Higher quality bar}
\end{Highlighting}
\end{Shaded}

\begin{center}\rule{0.5\linewidth}{0.5pt}\end{center}

\section{Thesis Alignment}\label{thesis-alignment}

This implementation directly addresses the thesis objectives:

\begin{enumerate}
\def\labelenumi{\arabic{enumi}.}
\tightlist
\item
  \textbf{Automate Security Requirements}: ✅ Converts high-level inputs
  to detailed requirements
\item
  \textbf{Standards Alignment}: ✅ Leverages OWASP, NIST, ISO 27001
\item
  \textbf{AI/ML Security}: ✅ Specialized agent for emerging threats
\item
  \textbf{Compliance}: ✅ Regulatory assessment integrated
\item
  \textbf{Quality Assurance}: ✅ Self-evaluation ensures completeness
\end{enumerate}

\begin{center}\rule{0.5\linewidth}{0.5pt}\end{center}

\section{Known Limitations \& Future
Work}\label{known-limitations-future-work}

\subsection{Current Limitations}\label{current-limitations}

\begin{enumerate}
\def\labelenumi{\arabic{enumi}.}
\tightlist
\item
  Sample security standards data (67 controls) - production would need
  complete standards
\item
  English language only
\item
  OpenAI dependency (no offline mode)
\item
  Single validation threshold for all dimensions
\end{enumerate}

\subsection{Future Enhancements}\label{future-enhancements}

\begin{enumerate}
\def\labelenumi{\arabic{enumi}.}
\tightlist
\item
  Complete security standards ingestion from official sources
\item
  Support for additional frameworks (SOC 2, CIS Controls)
\item
  Multi-language support
\item
  Custom validation weights per dimension
\item
  Integration with issue tracking systems (Jira, GitHub)
\item
  Real-time collaboration features
\end{enumerate}

\begin{center}\rule{0.5\linewidth}{0.5pt}\end{center}

\section{Testing \& Validation}\label{testing-validation}

\subsection{Recommended Testing
Approach}\label{recommended-testing-approach}

\begin{enumerate}
\def\labelenumi{\arabic{enumi}.}
\tightlist
\item
  \textbf{Unit Testing}: Test individual agent outputs
\item
  \textbf{Integration Testing}: Full flow with sample inputs
\item
  \textbf{Validation Testing}: Score analysis and feedback loop
\item
  \textbf{Standards Coverage}: Verify control retrieval accuracy
\end{enumerate}

\subsection{Sample Test Cases
Provided}\label{sample-test-cases-provided}

\begin{itemize}
\tightlist
\item
  Task management (general SaaS)
\item
  E-commerce (payment, GDPR)
\item
  Healthcare (HIPAA, PHI)
\end{itemize}

\begin{center}\rule{0.5\linewidth}{0.5pt}\end{center}

\section{Conclusion}\label{conclusion}

The system is fully implemented and ready for thesis evaluation. All
components work together to provide an end-to-end solution for automated
security requirements generation with self-evaluation capabilities.

The implementation demonstrates: - \textbf{Practical application} of
multi-agent systems - \textbf{Real-world value} for security engineering
- \textbf{Scalable architecture} for production deployment -
\textbf{Research contribution} to automated security requirements

\begin{center}\rule{0.5\linewidth}{0.5pt}\end{center}

\section{Contact \& Support}\label{contact-support}

For questions about this implementation, refer to: - \textbf{README.md}:
Complete usage documentation - \textbf{QUICKSTART.md}: Fast-start guide
- \textbf{Configuration}: \texttt{config.yaml}, \texttt{env.template}

\begin{center}\rule{0.5\linewidth}{0.5pt}\end{center}

\textbf{Implementation Complete}: October 14, 2025




\end{document}
